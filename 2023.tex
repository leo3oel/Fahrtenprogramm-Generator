\documentclass[12pt, a4paper]{report}
\usepackage{fontspec}
\setmainfont{CMU Serif}
\usepackage{marginnote}
\usepackage[german]{babel}
\usepackage{xurl}
\usepackage[colorlinks=true, linkcolor=blue, urlcolor=blue]{hyperref}
%Section auf größe von subsection setzen
\usepackage{titlesec}
\titleformat*{\section}{\large\bfseries}
%Grafik
\usepackage{float}
%\usepackage{pgf-pie}
\usepackage{pdfpages}
%\usepackage{pgf, tikz} %Tikz GRAPHICS

%Seitenlayout, Margins verkleinern
\usepackage[a4paper, headheight=1.75cm, top=3.2cm,bottom=30mm, left=4.5cm, right=1.5cm, marginparwidth=4cm, marginparsep=.4cm]{geometry} 
\usepackage{setspace}%Zeilenabstand verändern zB \onehalfspacing(Präambel)

\usepackage{amsmath}
\usepackage{csquotes}
\usepackage{graphicx} %Für Bilder
\usepackage{subcaption} %Für subfigure
\usepackage{siunitx}
%Bruchstrich
\sisetup{per-mode=fraction, output-decimal-marker = {,}, exponent-product=\cdot}
%Hier beliebige Einheiten hinzufügen
\DeclareSIUnit\year{Jahre}

%Kopf-Fußzeile
\usepackage{fancyhdr}
%\fancyhf{}
\pagestyle{fancy}
\fancyhead[L]{\Large{\textsc{Jahresprogramm}}}
\fancyhead[R]{\includegraphics[width=1.75cm]{logo.png}}
\renewcommand{\headrulewidth}{0.5pt}
\setlength{\headheight}{56.4pt}
\cfoot{\thepage}

%Paragraph Settings
\setlength{\parindent}{0cm}

%Titel
\title{Vorläufiges Fahrtenprogramm}
\date{}
%\author{Leonhard Schraut}
%Dokument

%%%%%%%%%%%%%%%%%%%%% Generated File %%%%%%%%%%%%%%%%%%%%%
\fancypagestyle{Wildwasser}{
    \fancyhead[L]{\Large{\textsc{Wildwasser}}}
    \fancyhead[R]{\includegraphics[width=1.75cm]{logo.png}}
    \renewcommand{\headrulewidth}{0.5pt}
    \cfoot{\thepage}
}

\begin{document}

\begingroup
    \hypersetup{hidelinks}
    \tableofcontents\thispagestyle{fancy}
\endgroup
\reversemarginpar

\section*{Bemerkungen}\paragraph{Allgemein}\mbox{}\\
Alle Termine sind bald auch auf unserem Google Kalender auf unserer Homepage zu finden.
\paragraph{Wildwasser}\mbox{}\\
Die Teilnahme an allen Fahrten erfolgt auf eigene Gefahr. Jede Haftung des Veranstalters und seiner Hilfspersonen für Personen- und Sachschäden ist ausgeschlossen, außer bei Vorsatz oder grober Fahrlässigkeit. Auch bei Wanderfahrten ist das Tragen von Schwimmwesten Pflicht. Ebenso ist auf, der Wassertemperatur angepasster, Kleidung zu achten.


Wildwasserfahrten nur mit entsprechender Ausrüstung: Wildwasserboot mit Auftriebskörpern und Bergegriffen, Schutzhelm, Wildwasserschwimmweste, Wurfsack, Kälteschutzkleidung für das Wasser (Neopren, Paddeljacke oder Trockenanzug), feste Paddelschuhe. Eine Haftung für Sachschäden oder Diebstahl ist ausgeschlossen.\chapter*{Wildwasser}
\thispagestyle{Wildwasser}
\addcontentsline{toc}{chapter}{\protect\numberline{}Wildwasser}
\pagestyle{Wildwasser}
\section*{Januar}\paragraph{Neujahrs Fahrt}\marginnote{08.01 \\ ab 09:00}
\begin{itemize}
    \item Ansprechpartner: Sebastian \href{mailto:wildwasser@kc-wuerzburg.de}{wildwasser@kc-wuerzburg.de}
    \item Fahrt auf Kleinfluss in der Nähe, z.B. Sinn 
\end{itemize}

\section*{April}\paragraph{Erster Mai Fahrt}\marginnote{28.04 - 01.05}
\begin{itemize}
    \item Ansprechpartner: Sebastian \href{mailto:wildwasser@kc-wuerzburg.de}{wildwasser@kc-wuerzburg.de}
    \item Ziel noch Unbekannt
\end{itemize}

\section*{Mai}\paragraph{BKV Lehrgang Wildalpen}\marginnote{18.05 - 21.05}
\begin{itemize}
    \item Ansprechpartner: Bernd Sachs \href{mailto:ww-breitensport@kanu-bayern.de}{ww-breitensport@kanu-bayern.de}
    \item Ansprechpartner KCW: Sebastian \href{mailto:wildwasser@kc-wuerzburg.de}{wildwasser@kc-wuerzburg.de}
    \item Schwierigkeit: WW I bis WW III
    \item Ort: Wildalpen
    \item Meldeschluss: 21.04.2023 (sehr schnell ausgebucht)
    \item Anmeldung: \url{www.kanu-bayern.de/Freizeitsport/WW-Breitensport/Termine/642/Wildwasser-Lehrgang-1/}
\end{itemize}

\paragraph{Pfingst Fahrt}\marginnote{25.05 - 29.05}
\begin{itemize}
    \item Ansprechpartner: Sebastian \href{mailto:wildwasser@kc-wuerzburg.de}{wildwasser@kc-wuerzburg.de}
    \item Ort: noch unbekannt
\end{itemize}

\section*{Juni}\paragraph{Durance/Seealpen}\marginnote{08.06 - 18.06}
\begin{itemize}
    \item Ansprechpartner: Sebastian \href{mailto:wildwasser@kc-wuerzburg.de}{wildwasser@kc-wuerzburg.de}
    \item Schwierigkeit: WW II bis WW IV
    \item Bis 13.06 Durance, anschließend weiterfahrt z den Seealpen
\end{itemize}

\section*{September}\paragraph{Haiming}\marginnote{29.09 - 03.10}
\begin{itemize}
    \item Ansprechpartner: Sebastian \href{mailto:wildwasser@kc-wuerzburg.de}{wildwasser@kc-wuerzburg.de}
    \item Schwierigkeit: WW II bis WW IV
    \item Ort: Haiming
    \item Gepaddelt werden: Ötz, Inn und Sanna
\end{itemize}

\section*{Juli}\paragraph{BKV Lehrgang Augsburg}\marginnote{14.07 - 16.07}
\begin{itemize}
    \item Ansprechpartner: Bernd Sachs \href{mailto:ww-breitensport@kanu-bayern.de}{ww-breitensport@kanu-bayern.de}
    \item Ansprechpartner KCW: Sebastian \href{mailto:wildwasser@kc-wuerzburg.de}{wildwasser@kc-wuerzburg.de}
    \item Technik-Lehrgang am Augsburger Eiskanal
    \item Weitere Informationen und Anmeldung: \url{www.kanu-bayern.de/Freizeitsport/WW-Breitensport/Termine/643/Wildwasser-Lehrgang-2/}
\end{itemize}

\section*{September}\paragraph{BKV Lehrgang Lechtal}\marginnote{16.09 - 17.09}
\begin{itemize}
    \item Ansprechpartner: Bernd Sachs \href{mailto:ww-breitensport@kanu-bayern.de}{ww-breitensport@kanu-bayern.de}
    \item Ansprechpartner KCW: Sebastian \href{mailto:wildwasser@kc-wuerzburg.de}{wildwasser@kc-wuerzburg.de}
    \item Schulung für Erwachsene sowie Familien (incl. Kinder)
    \item Schwierigkeit: WW I bis WW III
    \item Ort: Vorderhornbach
    \item Weitere Informationen und Anmeldung: \url{www.kanu-bayern.de/Freizeitsport/WW-Breitensport/Termine/644/Wildwasser-Lehrgang-3-Schulung-fuer-Erwachsene-sowie-Familien-incl-Kinder/}
\end{itemize}


\end{document}